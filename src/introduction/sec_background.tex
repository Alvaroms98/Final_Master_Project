\section{Background}

High Performance Computing (HPC) has revolutionized the field of engineering simulations by enabling faster and more accurate analyses of complex systems. HPC involves the use of advanced computing technologies, such as parallel processing, to perform computational simulations on large-scale systems.

Engineering simulations can involve a wide range of applications, from structural analysis and fluid dynamics to electromagnetic and thermal analysis. HPC allows engineers to analyze these systems in greater detail and with greater accuracy than ever before, enabling them to make more informed decisions about design, performance, and safety.

One of the key benefits of HPC is the ability to perform simulations on much larger and more complex systems than would be possible with traditional computing methods. For example, HPC can be used to simulate the behavior of entire buildings or bridges, or to model the flow of fluids through complex geometries such as aircraft wings or turbine blades.

Another key benefit of HPC is the ability to perform simulations quicker than before. This enables engineers to perform more iterations of a design in a shorter amount of time, allowing them to optimize the design reducing the amount of time to wait for result analysis. This translates into cost reduction and safety improvement in engineering projects. By using validated simulations\footnote{A validated simulation is a computer-based model that has been tested and verified to accurately represent the real-world system or process it is intended to simulate.}, engineers can identify potential problems and make design changes before construction or deployment, reducing the risk of costly and dangerous failures.

Programming plays a critical role in HPC. The effectiveness and efficiency of an application depends largely on how well the software is designed, coded and optimized. To make the most of HPC applications, the programmers/engineers must be skilled in parallel programming, concurrent systems, input/output optimization, deep understanding in memory management, between others. They must be knowledgeable about the specific hardware and software environment in which the application will be run, and they must be able to optimize the application to make the best use of available resources.

The thesis is focused on the use of HPC for combustion simulations, specifically on speeding up the chemical calculation by parallelizing the Ordinary Differential Equation (ODE) systems formed by the chemical species of the fuel and the temperature. Combustion is the process of burning fuel in the engine, which produces energy to propel the engine. The combustion process in gas turbine engines is complex and involves a variety of physical phenomena, such as chemical reactions, fluid dynamics, and heat transfer.

Chemical reactions occur between the fuel and air, resulting in the formation of combustion products, which in turn generate heat. Fluid dynamics describe the movement of gases through the engine, which is influenced by factors such as temperature, pressure, and velocity. Heat transfer refers to the transfer of heat energy between different parts of the engine, which can affect engine performance and safety.

HPC is a technology that allows for the use of advanced computing resources, such as parallel processing, to perform simulations of complex systems. In this case, HPC is used to simulate the combustion process in gas turbine engines with greater accuracy and detail than traditional computing methods. By using HPC, the author of the Master's Thesis aims to gain insights into the behavior of combustion processes in gas turbine engines, including the performance, emissions, and safety of the engine. The use of HPC is expected to enable faster and more efficient simulation of these complex physical phenomena, ultimately improving the design and optimization of gas turbine engines.